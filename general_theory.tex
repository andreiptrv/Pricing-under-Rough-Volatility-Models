\section{Realized volatility}
    Consider a stochastic volatility model 
    \begin{equation}\label{model:SVM}
        dS_t = \mu_t S_t dt + \sigma_t S_tdW_t,
    \end{equation}
    where $S_t$ is an asset price process, and $\sigma_t$ is a stochastic volatility process representing a so-called \emph{spot volatility}.
    Spot volatility, in fact, is not observable in the market, therefore, we should estimate it somehow.

    \begin{definition}
        The \emph{realized variance} of a price process $S$ over time interval $[t, t + \delta]$ sampled along the time partition $\pi^n$
        is defined as 
        \begin{equation}\label{def:RV}
            RVar_{t, t + \delta}(\pi^n) = \sum_{\pi^n \cap [t, t + \delta]} \left(\log S_{t^n_{i+1}} - \log S_{t^n_{i}}\right)^2,
        \end{equation}
        and \emph{realized volatility} is defined as
        \begin{equation}\label{def:RVol}
            RV_{t, t + \delta}(\pi^n) = \sqrt{\sum_{\pi^n \cap [t, t + \delta]} \left(\log S_{t^n_{i+1}} - \log S_{t^n_{i}}\right)^2}.
        \end{equation}
    \end{definition}
    As pointed out in \cite{rvol_wsm}, realized volatility has some limitations:
    \begin{enumerate}
        \item The volume of data used influences the end results during the calculation of realized volatility. At least 20 observations are statistically required to calculate a valid value of realized volatility. Therefore, realized volatility is better used to measure longer-term price risk in the market ($\sim$ 1 month or more).
        \item Realized volatility calculations are directionless. i.e., it factors in upward and downward trends in price movements.
        \item It is assumed that asset prices reflect all available information while measuring volatility.
    \end{enumerate}


    \begin{definition}
        Let $S$ satisfy \eqref{model:SVM}. Then the integrated variance is defined as
        \begin{equation}\label{def:IV}
            IVar_t = \int_{0}^{t} \sigma_s^2 ds.
        \end{equation}
    \end{definition}
    It has been shown many times (e.g. \cite{Barndorff-Nielsen2002}) and mentioned in \cite{Cont2022} that the realized variance converges 
    in probability to the integrated variance as sampling frequency increases for all assets satistying the equation 
    \eqref{model:RFSVasset} (i.e. stochastic volatility models).
    \begin{proposition} 
        As time partition scale of $\pi^n$ tends to $0$, $RV_{t, t + \delta}(\pi^n) \approx \sqrt{\delta} \sigma_t$, i.e. $RV_{t, t + \delta}/\sqrt{\delta}$ could be considered as a consistent estimator of the spot volatility.
    \end{proposition}

\section{Fractional Stochastic Processes}
    \begin{definition}
        The \emph{fractional Brownian motion} $(W_t^H)_{t\in \mathbb{R_+}}$ with Hurst 
        parameter $H \in (0, 1)$ is a Gaussian process with the following properties:
        \begin{enumerate}
            \item $W_0^H = 0$,
            \item $\mathbb{E}\left[W_t^H\right] \equiv 0$,
            \item $\mathbb{E}\left[W_s^H W_t^H\right] = \frac{1}{2} \left(t^{2H} + s^{2H} - |t-s|^{2H}\right)$.
        \end{enumerate}
    \end{definition}

    \begin{definition}
        A stationary fOU process $X_t$ is defined as the stationary solution of the stochastic differential equation
        \begin{equation}
            dX_t = \nu dW^H_t - \alpha (X_t - m)dt,
        \end{equation}
        where $m \in \mathbb{R}$ and $\nu$ and $\alpha$ are positive parameters, see \cite{Cheridito2003}.
    \end{definition}

    \begin{definition}
        Let us define $\Delta _{h}f(x):=f(x-h)-f(x)$ and let us define the modulus of continuity by
        \begin{equation}
            \omega _{p}^{2}(f,t)=\sup _{|h|\leq t}\left\|\Delta _{h}^{2}f\right\|_{p}.
        \end{equation}
        Let $n$ be a non-negative integer and $s = n + \alpha$ with $\alpha \in \left(0, 1\right]$. 
        The \emph{Besov space} $B^{s}_{p,q}(\mathbb{R})$ contains all functions $f\in W^{n, p}(\mathbb{R})$ such that
        \begin{equation}
            \int_0^\infty \left|\frac{ \omega^2_p (f^{(n)}, t)} {t^{\alpha} }\right|^q \frac{dt}{t} < \infty.
        \end{equation}
        The Besov space $B_{p,q}^{s}(\mathbb{R})$ is a normed space with the standard norm defined as
        \begin{equation}
            \left\|f\right\|_{B_{p,q}^{s}(\mathbf {R} )}^q=\|f\|_{W^{n,p}(\mathbb {R} )}^{q}+\int _{0}^{\infty }\left|{\frac {\omega _{p}^{2}(f^{(n)}, t)}{t^{\alpha }}}\right|^{q}{\frac {dt}{t}}.
        \end{equation}
    \end{definition}

    \subsection{What is a long-memory process?}
        \begin{definition}
            A process $X_t$ is said to have a long memory, if 
            \begin{equation}
                \sum_{k=0}^{\infty}\cov\left[X_1 , X_{k} - X_{k-1}\right] = \infty.
            \end{equation}
        \end{definition}

        In particular, the fractional Brownian motion with $H > \frac{1}{2}$ is a long-memory process.
        Long-memory of the stochastic volatility process in stochastic volatilty models framework used to be a 
        widely-accepted stylized fact \cite{Breidt1998,ComteRenault1998,Comte1996,Ding1993}.

\section{Normality Statistical Tests}\label{sec:statisticaltest}
    In the following, $x_i$ denotes a sample of $n$ observations, $g_1$ and $g_2$ are the sample 
    skewness and excessed kurtosis, $\mu_j$'s are the $j$-th sample central moments, and $\overline{x}$ is the 
    sample mean. 
    \subsection{D'Agostino's K-squared test}
        The sample skewness and kurtosis are defined as
        \begin{align}
            & g_1 = \frac{ m_3 }{ m_2^{3/2} } = \frac{\frac{1}{n} \sum_{i=1}^n \left( x_i - \overline{x} \right)^3}{\left( \frac{1}{n} \sum_{i=1}^n \left( x_i - \overline{x} \right)^2 \right)^{3/2}}\ , \\
            & g_2 = \frac{ m_4 }{ m_2^{2} }-3 = \frac{\frac{1}{n} \sum_{i=1}^n \left( x_i - \overline{x} \right)^4}{\left( \frac{1}{n} \sum_{i=1}^n \left( x_i - \overline{x} \right)^2 \right)^2} - 3\ .
        \end{align}
        Let
        \begin{equation}
            Z_1(g_1) = \delta \operatorname{asinh}\left( \frac{g_1}{\alpha\sqrt{\mu_2}} \right),
        \end{equation}
        where constants $\alpha$ and $\delta$ are computed as
        \begin{align}
            & W^2 = \sqrt{2\gamma_2 + 4} - 1, \\
            & \delta = 1 / \sqrt{\ln W}, \\
            & \alpha^2 = 2 / (W^2-1),
        \end{align}
        and
        \begin{equation}
            Z_2(g_2) = \sqrt{\frac{9A}{2}} \left\{1 - \frac{2}{9A} - \left(\frac{ 1-2/A }{ 1+\frac{g_2-\mu_1}{\sqrt{\mu_2}}\sqrt{2/(A-4)} }\right)^{\!1/3}\right\},  
        \end{equation}
        where
        \begin{align}
            & A = 6 + \frac{8}{\gamma_1} \left( \frac{2}{\gamma_1} + \sqrt{1+4/\gamma_1^2}\right), \\
            & \mu_1(g_2) = - \frac{6}{n+1}, \\
            & \mu_2(g_2) = \frac{ 24n(n-2)(n-3) }{ (n+1)^2(n+3)(n+5) }, \\
            & \gamma_1(g_2) \equiv \frac{\mu_3(g_2)}{\mu_2(g_2)^{3/2}} = \frac{6(n^2-5n+2)}{(n+7)(n+9)} \sqrt{\frac{6(n+3)(n+5)}{n(n-2)(n-3)}},  \label{eq:gamma1}\\
            & \gamma_2(g_2) \equiv \frac{\mu_4(g_2)}{\mu_2(g_2)^{2}}-3 = \frac{ 36(15n^6-36n^5-628n^4+982n^3+5777n^2-6402n+900) }{ n(n-3)(n-2)(n+7)(n+9)(n+11)(n+13) }\label{eq:gamma2}.
        \end{align}
        The analytical expressions for skewness and kurtosis \eqref{eq:gamma1} - \eqref{eq:gamma2} were derived by E. Pearson in \cite{Pearson1931}.
        \begin{definition}
            The \emph{D'Agostino--Pearson} statistic is defined as 
            \begin{equation}
                K^2 = Z_1(g_1)^2 + Z_2(g_2)^2
            \end{equation}
            $H_0$: the sample is normally distributed.
        \end{definition}

        \begin{remark}
            The $K^2$ statistic is able to detect deviations from both skewness and kurtosis. 
            If the null hypothesis is true, then the test statistic has the $\chi^2$ 
            distribution with $2$ degrees of freedom. 
        \end{remark}

    \subsection{Shapiro--Wilk test}
        \begin{definition}
            The Shapiro--Wilk test statistic is defined as
            \begin{equation}
                W = \frac{\left(\sum_{i=1}^n a_i x_{(i)}\right)^2 }{\sum_{i=1}^n (x_i-\overline{x})^2},
            \end{equation}
            where
            \begin{equation*}
                (a_1, \dots, a_n) = \frac{m^{T} V^{-1}}{C}, \quad C = \| V^{-1} m \| = (m^{T} V^{-1}V^{-1}m)^{1/2},
            \end{equation*}
            and $m = (m_1, \dots, m_n)^{T}$ is a mean of order statistic from a normally distributed sample, $V$ \text{ is the covariance matrix of those normal order statistics}
            $H_0$: the sample is normally distributed.
        \end{definition}

        \begin{remark}
            The $W$ statistic has no distinguishable name, and the cutoff values are calculated numerically by Monte-Carlo simulation.
        \end{remark}

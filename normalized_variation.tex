\section{Sample normalized variation as a measure of roughness}
    Let us consider a sequence of partitions $\pi^n$ of $\left[0, T\right]$ with 
    $\left|\pi^n\right|:=\max_{t_i^n \in \pi^n} (t_{i+1}^n - t_i^n) \to 0$. 
    \begin{definition}
        A function $x \in C[0, T]$ is said to have the finite $p$-th variation along the sequence of partitions $\pi^n$ 
        if there exists a continious increasing function $\left[x\right]_{\pi}^{(p)}$ such that for all subpartitions $\tilde\pi^n(t) = \pi^n \cap [0, t]$
        \begin{equation}
            \sum_{t_i^n \in \tilde\pi^n(t)} \left|x(t_{i+1}^n) - x(t_i^n)\right|^p \to \left[x\right]_{\pi}^{(p)}(t), \quad n\to \infty,
        \end{equation}
        and the set of all functions having finite $p$-th variation along $\pi$ we denote $V_\pi^p$.
    \end{definition}


    \begin{figure}
        \centering
        \includegraphics[width=\linewidth]{fig/W Stat Illustration.pdf}
        \caption{The $W$ statistic illustration}
    \end{figure}

\section{Roughness estimation of Monte-Carlo simulations}

    \subsection{Brownian and fractional Brownian motion}
        We shall test our method on those processes, whose roughness is well-known.
        \subsubsection{Brownian motion}
            \begin{figure}
                \centering
                \includegraphics[width=\linewidth]{fig/Histogram for Roughness of MC-d Brownian Motion.pdf}
                \caption{Histogram for roughness of Brownian motion}
            \end{figure}

        \subsubsection{Fractional Brownian motion (Davies-Harte method)}
            We considered four Hurst parameters for simulation: $0.15$, $0.35$, $0.55$, and $0.75$. We used the Davies-Harte method of generating 
            the fBm since this one is widely accepted as the most precise.
            \begin{figure}
                \centering
                \includegraphics[width=\linewidth]{fig/Histogram. Roughness of Fractional Brownian Motion with H = 0.15, 100 Monte-Carlo Simulations.pdf}
                \caption{Histogram for roughness of fractional Brownian motion}
            \end{figure}
            \begin{figure}
                \centering
                \includegraphics[width=\linewidth]{fig/Histogram. Roughness of Fractional Brownian Motion with H = 0.35, 100 Monte-Carlo Simulations.pdf}
                \caption{Histogram for roughness of fractional Brownian motion}
            \end{figure}
            \begin{figure}
                \centering
                \includegraphics[width=\linewidth]{fig/Histogram. Roughness of Fractional Brownian Motion with H = 0.55, 100 Monte-Carlo Simulations.pdf}
                \caption{Histogram for roughness of fractional Brownian motion}
            \end{figure}
            \begin{figure}
                \centering
                \includegraphics[width=\linewidth]{fig/Histogram. Roughness of Fractional Brownian Motion with H = 0.75, 100 Monte-Carlo Simulations.pdf}
                \caption{Histogram for roughness of fractional Brownian motion}
            \end{figure}

    \subsection{Heston stochastic volatility model}

    \begin{figure}
        \centering
        \includegraphics[width=\linewidth]{fig/Histogram. Roughness of Heston Monte-Carlo Simulations (Instantaneous volatility).pdf}
        \caption{Histogram for roughness of Heston SVM}
    \end{figure}
    \begin{figure}
        \centering
        \includegraphics[width=\linewidth]{fig/Histogram. Roughness of Heston Monte-Carlo Simulations (Realized volatility).pdf}
        \caption{Histogram for roughness of Heston SVM}
    \end{figure}


\section{Roughness estimation of real-market data}

    Let us estimate the roughness of real-market data. 
    We are using the same Bloomberg data from Table \ref{table:hurst_est} (but without indexes).

    \begin{table}[h]
        \centering
        \begin{tabular}{|c|c|c|c|c|c|}
            \hline
            Ticker &  Roughness Index\\\hline
            \hline
            YNDX RX Equity & 0.372691\\\hline
            SBER RX Equity & 0.313109\\\hline
            VTBR RX Equity & 0.304677\\\hline
            MOEX RX Equity & 0.295378\\\hline
            LKOH RX Equity & 0.301795\\\hline
            GAZP RX Equity & 0.316125\\\hline
            FIVE RX Equity & 0.284704\\\hline
            \hline
            OGZD LI Equity & 2.968608\\\hline
            VTBR LI Equity & 0.306763\\\hline
            SBER LI Equity & 1.176616\\\hline
            LKOD LI Equity & 0.306061\\\hline
        \end{tabular}
        \caption{Roughness index estimation}
        \label{tab:roughness_index}
    \end{table}

    As we can see, modelless estimation of roughness differs from Hurst parameter (as Monte-Carlo simulations predicted).

\documentclass[]{amsart}
%\usepackage[a4paper, margin=0.3in]{geometry}

\usepackage[T1,T2A]{fontenc}
\usepackage[utf8]{inputenc}
\usepackage{amsmath,amsthm,amssymb}
\usepackage{mathtext}
\usepackage{amsthm}
\usepackage{graphicx}
\usepackage{cmap}
\usepackage{bbm}
\usepackage{tikz}

% Types

\newtheorem{theorem}{Theorem}
\newtheorem{lemma}{Lemma}
\newtheorem{prop}{Proposition}
\newtheorem{col}{Corrolary}

\theoremstyle{definition}
\newtheorem{definition}{Definition}

\theoremstyle{definition}
\newtheorem{example}{Example}

\theoremstyle{definition}
\newtheorem*{nb}{NB}


\newcommand{\hm}[1]{#1\nobreak\discretionary{}{\hbox{\ensuremath{#1}}}{}}
\newcommand{\nn}{\nonumber}
\newcommand{\E}{\mathbb{E}}


\DeclareMathOperator{\sgn}{sign}
\DeclareMathOperator*{\var}{var}   
\DeclareMathOperator*{\Var}{Var}     
\DeclareMathOperator*{\cov}{cov}

\newcommand{\1}{\mathbbm{1}} 
\newcommand{\Co}{\mathbb{C}}
\newcommand{\R}{\mathbb{R}}
\newcommand{\N}{\mathbb{N}}
\renewcommand{\P}{P}
\newcommand{\eps}{\varepsilon}

\numberwithin{equation}{section}
\numberwithin{theorem}{section}
\numberwithin{lemma}{section}
\numberwithin{prop}{section}
\numberwithin{col}{section}
\numberwithin{definition}{section}
\numberwithin{example}{section}

\usepackage{hyperref}
\hypersetup{
	colorlinks=true,
	linkcolor=blue,
	filecolor=magenta,      
	urlcolor=cyan,
}

\date{}
\title{Cheatsheet for Mathematical Finance Students}
\author{Vega Institute Foundation}
